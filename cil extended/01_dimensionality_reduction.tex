\part{Dimensionality Reduction}
Select the \emph{most interesting} dimensions. 

\section{Intrinsic Dimensionality}
\subsection*{Pairwise Distances}
Assume components of data $x=(x_1,\ldots,x_D)^T \in \R^D$ are i.i.d. Gaussian distributed:
\begin{align*}
    x_d \sim \normal(0,1) \implies x_d - y_d \sim \normal(0,2).
\end{align*}
Using $\chi^2$-distribution:
\begin{align*}
    {1\over 2}(x_d-y_d)^2 \sim \chi^2(1),
\end{align*}
and extending to $D$ dimensions:
\begin{align*}
    {1\over 2} \sum_{d=1}^D (x_d-y_d)^2 \sim \chi^2(D) = \Gamma\left( {D\over 2}, 2\right)\\
    \text{Recall: } \forall z,k, \theta > 0,\ \Gamma(z;k,\theta) = {\theta^k\over \Gamma(k)} y^{k-1} e^{-\theta z} 
\end{align*}
Hence, the dimension-normalised squared distance is:
\begin{align*}
    {1\over D} \sum_{d=1}^D (x_d-y_d)^2 \sim \Gamma\left({D\over 2}, {4\over D}\right)
\end{align*}
is Gamma distributed with mean $2$ and variance ${8 \over D}$. 

$\Gamma\left({D\over 2}, {4\over D}\right)$  tends towards normality with shrinking width for large $D$. Therefore, most points have \emph{constant} pairwise distances in this limit.

\section{Principal Component Analysis}
Objectives of PCA:
\begin{enumerate}
\item Minimise error $\norm{x_n-\tilde{x}_n}$ of point $x_n$ and its approximation $\tilde x_n$.
\item Reveal "interesting" information: maximise \emph{variance}.
\end{enumerate}
Both objectives are show to be formally equivalent.

Consider a set of observations $\{x_n\},\ n=1, \ldots, N$ and $x_n \in \R^D$.
\begin{description}
\item[Goal] Project data onto $K<D$ dimensional space while maximising variance of the projected data.
\item[For $\mathbf{K=1}$]  Define direction of projection as $u_1$. Set $\norm{u_1}_2 =1$ (only the direction of the projection is important.
\end{description}
\subsection{Statistics of Projected Data}
\paragraph{Original Data}
\begin{description}
\item[Mean] is given by the sample mean $\bar x$.
\item[Covariance] of the Data:
\begin{align*}
    \Sigma   = {1\over N} \sum_{n=1}^N (x_n - \bar x)(x_n - \bar x)^T
\end{align*}
\end{description}
\paragraph{Projected Data}
\begin{description}
\item[Mean] is given by: $u_1^T\bar x$.
\item[Variance] is given by:
\begin{align*}
    {1\over N} \sum_{n=1}^N \left\{ u_1^T x_n-u_1^T \bar x\right\}^2 
        &= {1\over N} \sum_{n=1}^N \left\{u_1^T (x_n-\bar x)\right\}^2\\
        &= {1\over N} \sum_{n=1}^N u_1^T (x_n - \bar x)(x_n-\bar x)^T u_1\\
        &= u_1^T \Sigma u_1.
\end{align*}
\end{description}

\subsection{Maximisation Problem}
These statistics now can be fed into a maximisation problem:
\begin{align*}
    \max_{u_1}\ u_1^T \Sigma u_1
\end{align*}
such that $\norm{u_1}_2 = 1$.

Writing the Lagrangian results in in:
\begin{align*}
    \mathcal L:= u_1^T \Sigma u_1 + \lambda_1(1-u_1^Tu_1). 
\end{align*}
Setting ${\delta \over \delta u_1}\mathcal L \overset{!}{=} 0$ results in:
\begin{align*}
    \Sigma u_1 = \lambda_1 u_1
\end{align*}

We observe that $u_1$ is an \emph{eigenvector} of $\Sigma$ and $\lambda_1$ it's associated \emph{eigenvalue}. Furthermore $\lambda_1$ is also the variance of the projected data:
\begin{align*}
    \lambda_1 = u_1^T \Sigma u_1
\end{align*}
\subsubsection{Second principal direction}
The second principal direction can be obtained by maximising the variance $u_2^T \Sigma u_2$, subject to $\norm{u_2}_2 = 1$ and $u_2^Tu_1 = 0$:
\begin{align*}
    \mathcal L = u_2^T \Sigma u_2 + \lambda_2\left( 1- u_2^Tu_2\right) +\nu \left(u_2^Tu_1\right).
\end{align*}
The maximum i found by setting ${\delta \mathcal L \over \delta u_2} \overset{!}{=} 0$:
\begin{align*}
    2\Sigma u_2 - 2\lambda_2 u_2+\nu u_1 = 0.
\end{align*}
Because of the orthogonality between $u_2$ and $u_1$ we observe that $u_2$ contains no component of $u_1$ and hence $\nu = 0$. We get:
\begin{align*}
    \Sigma u_2 = \lambda_2 u_2.
\end{align*}
We observe that $u_2$ is an eigenvector of $\Sigma$ with the second largest eigenvalue of $\lambda_2$.
\subsection{Solution: Eigenvalue Decomposition}
Hence we see that the eigenvalue decomposition of the covariance matrix
\begin{align*}
    \Sigma = U\Lambda U^T
\end{align*}
contains all relevant information. 

For a projection space of size $K\leq D$ we choose the $K$ eigenvectors $\{u_1,\ldots, u_k\}$ with the largest associated eigenvalues $\{\lambda_1, \ldots,\lambda_2\}$.

\subsection{Error Formulation}
We define an \emph{orthonormal} basis $\{u_d\}$, $d=1,\ldots, D$ of $\R^D$. The scalar projection of $x_n$ onto $u_d$ (magnitude) is given by:
\begin{align*}
    z_{n,d} = x_n^Tu_d.
\end{align*}
The associated projection onto $u_d$ amounts to $z_{n,d}u_d$. Therefore, each data point can be represented in the basis by:
\begin{align*}
    x_n = \sum_{d=1}^D z_{n,d}u_d=\sum_{d=1}^D\left(x_n^Tu_d\right) u_d.
\end{align*}

\emph{Restricted representation} using $K<D$ basis vectors can be written as:
\begin{align*}
\tilde x_n = \sum_{d=1}^K a_{n,d}u_d + \sum_{d=K+1}^D b_d u_d,
\end{align*}
where $b_d$ does not depend on the data point $x_n$.

The approximation error can be represented by:
\begin{align*}
 J(\{a_{n,d}\}, \{ b_d\}) &= {1\over N} \sum_{n=1}^N \norm{x_n-\tilde x_n}_2^2\\
\text{Minimisation of $J$ w.r.t.} a_{n,d} &= x_n^T\\
\text{Minimisation of $J$ w.r.t.} b_d &= \bar x^T u_d
\end{align*}
The displacement can be obtained by resubsittuing $a_{n,d}$ and $b_d$:
\begin{align*}
x_n - \tilde x_n = \sum_{d=K+1}^D \left\{\left(x_n-\bar x\right)^Tu_d\right\}u_d.
\end{align*}
We observe that the displacement vector is orthogonal to the principal space!

Resubstituting the displacement into the error criterion leads to:
\begin{align*}
    J= {1\over N} \sum_{n=1}^N \sum_{d=K+1}^D \left(x_n^T u_d - \bar x^T u_d\right)^2 = \sum_{d=K+1}^D u_d^T \Sigma u_d
\end{align*}

\subsection{Matrix viewpoint}
The data can be represented as matrix:
\begin{align*}
    X= [x_1,\ldots, x_n,\ldots, x_N]
\end{align*}
The corresponding zero-centered data is:
\begin{align*}
    \bar X = X-M,
\end{align*}
where $M=\underbrace{[\bar x, \ldots, \bar x]}_{N\ \text{times}}$.

Compute the projection of $\bar X$ on $U_k = [u_1,\ldots, u_K]$  with:
\begin{align*}
    \underbrace{\bar Z_K}_{K\times N} = 
        \underbrace{U_K^T}_{K\times D} \cdot \underbrace{\bar X}_{D \times N}.
\end{align*}

To approximate $\bar X$, we return to the original basis:
\begin{align*}
    \tilde{\bar X} = U_K\cdot \bar Z_K.
\end{align*}

For $K=D$  we obtain a perfect reconstruction.

\subsection{Computation}
First compute the \emph{empirical mean}:
\begin{align*}
\bar x = {1\over N} \sum_{n=1}^N x_n
\end{align*}
Then \emph{center the data} by subtracting the mean from each sample:
\begin{align*}
\bar X = X-M,
\end{align*}
where $M=\underbrace{[\bar x, \ldots, \bar x]}_{N\ \text{times}}$.
Now compute the \emph{Covariance matrix}:
\begin{align*}
\Sigma = {1\over N} \sum_{n=1}^N (x_n-\bar x)(x_n-\bar x)^T = {1\over N} \underbrace{\bar X\bar X^T}_{\text{Scatter Matrix }\mathbf S}.
\end{align*}
$\Sigma$ is \emph{symmetric}. 

Now the \emph{Eigenvalue decomposition} can be computed:
\begin{align*}
\Sigma = U\Lambda U^T,
\end{align*}
where $\Lambda = diag[\lambda_1, \ldots, \lambda_D]$, such that $\lambda_1 \geq \lambda_2 \geq \ldots \geq \lambda_D$ with orthonormal eigenvectors.

\begin{description}
\item[Transformation] the data can be transformed on to the new basis of $K$ dimensions:
\begin{align*}
    \tilde{\bar Z} = U_K^T \bar X,
\end{align*}
$\bar Z \in \R^{K\times N}$: We obtain a dimension reduction of the data.
\item[Reconstruction] Go back to the original basis by computing
\begin{align*}
    \tilde{\bar X} &= U_K \bar Z\\
    \tilde X &= \tilde{\bar X}+M
\end{align*}


\end{description}



