\section{Non-Negative Matrix Factorisation}

Problem:
\begin{itemize}
    \item Given: Corpus of text documents such as web pages
    \item Goal: Find a low-dimensional representation of these documents. 
\end{itemize}

\paragraph{Document Representation}
\begin{description}
\item[Vocabulary] Every semantically "useful"word in a language. This excludes all the stop words (the, is, at, which, etc...). Stemming reduces the text to it's root form: Cats, catlike, catty, etc. all map to the same word "cat".


$D$ denotes the size of the vocabulary.

\item[Document] \emph{Bag of words}-model: Ordering of the words in a document is ignored. The document is represented by a vector of length $D$ with frequencies/counts of different words. This vector is usually very sparse.
\end{description}

\subsection{Matrix view} Here $N$ denotes the number of documents and $K$ denotes the number of "clusters" with $D$ being the vocabulary size.
\begin{itemize}
    \item $X \in \R_+^{D\times N}$ denotes the document-term matrix which stores the word counts for each document:
        \begin{align*}
            X = [x_1,\ldots, x_N]
        \end{align*}
        $x_{dn}$: Frequency of the $d$-th word in the $n$-th document.
    \item We study a non-negative matrix factorisation (NMF) of the document matrix $X$:
        \begin{align*}
            X\approx UZ
        \end{align*}
        with $U\in \R_+^{D\times K}$ and $Z\in \R_+^{K\times N}$, with $\R_+ := [0,\ldots, \infty)$ . 
\end{itemize}
\subsection{Methods}
\subsubsection{Full Singular Value Decomposition}
A singular value decomposition can be used to decompose the document word matrix:
\begin{align*}
    X = U\Sigma V^T.
\end{align*}
However the procured $U$ and $V$ are not guaranteed to be non-negative. 

\subsubsection{Classic Latent Semantic Indexing (LSI)}
The LSI method uses a partial SVD
\begin{align*}
    \tilde X = U\tilde \Sigma V^T \approx X.
\end{align*}
With $\tilde \Sigma$ having all but the largest $K$ singular values set to zero. 

\paragraph{Mapping function} A query $x\mapsto (U\tilde \Sigma)^T x$
 can now be mapped by the query can now be compared/queried with an inner product:
\begin{align*}
\langle \overline x_1 , \overline x_2\rangle = 
    \left( 
        \left(
            U\tilde \Sigma
        \right)^T x_1
     \right)^T
    \left( U\tilde \Sigma \right)^T x_2 = x_1 U\tilde \Sigma^2 U^T x_2.
\end{align*}

\subsubsection{Probabilistic Latent Semantic Indexing (pLSI)}
Fixes some shortcomings of LSI:
\begin{itemize}
    \item Probabilistic
    \item Takes non-negativity into account. 
\end{itemize}

\paragraph{Generative model} In order to generate a tuple $(document,\ word)$:
\begin{itemize}
\item Sample a document according to $P(document)$.
\item Sample a word according to $P(word|document)$.
\item Assume a factorisation:
    \begin{align*}
        P(word|document) = \sum_{topic} P(word|topic) P(topic|document)
    \end{align*}
    Which assumes a conditional independence of word and document given the topic.
\end{itemize}

The joint distribution of a document and a word is therefore:
\begin{align*}
    P(word,\ document) = P(word|document)P(document).
\end{align*}

\paragraph{Matrix Factorisation View} In a first step normalise the elements of $X$ so that they can be interpreted as probabilities:
\begin{align*}
    P(d-th\ word,\ n-th\ document) = {x_{dn}\over \sum_{d',\ n'} x_{d'n'}}.
\end{align*}

pLSI can be understood as a matrix factorisation of the form
\begin{align*}
    X\approx UZ,
\end{align*}
with $U\in R_+^{D\times K}$ and $Z\in R_+^{K\times N}$. Additionally we have the constraints:
\begin{align*}
    \sum_{d=1}^D u_{dk} = 1\ \forall k\qquad \text{and}\qquad \sum_{k,n} z_{kn} =1.
\end{align*}

\paragraph{Parameter Estimation} We want to maximise the likelihood of the data under the mode. Data: The normalised occurrence $X$ with the model:
\begin{align*}
    P(word,\ document) = \sum_{topic} P(word|topic) P(topic,\ document).
\end{align*}
Due to the constraints on $U$ and $Z$ the pLSI model can be equivalently written as:
\begin{align*}
P(d-th\ word,\ n-th\ document) = x_{dn} = (UZ)_{dn}.
\end{align*}

\paragraph{Kullback-Leibler Divergence} between two discrete probability distributions $P(x)$ and $Q(x)$:
\begin{align*}
    D_{KL} (P||Q) = \sum_{x\in \mathcal X} P(x) \log \left(
        {P(x)\over Q(x)}
        \right)
\end{align*}
Properties:
\begin{align*}
    D_{KL} (P||Q) &\geq 0.\\
    D_{KL} (P||Q) &= 0 &\text{iff $P$ and $Q$ are identical.}\\
    D_{KL}(P||Q) &\neq D_{KL}(Q||P)!
\end{align*}

The KL-divergence is not symmetric and therefore is not a metric/distance.

\paragraph{Kullback-Leibler Divergence applied for pLSI} maximising the likelihood of the data under the model is the same as minimising the KL-divergence of $X$ and $UZ$:
\begin{align*}
    \min_{U,Z} &\sum_{d=1}^D\sum_{n=1}^N x_{dn} \log\left({x_dn\over (UZ)_{dn}}\right)\\
    \text{s.t.}\ & \sum_{d=1} ^D u_{dk} = 1\forall k,\\
    &\sum_{k,n} z_{kn} =1\\
    &u_{dk} \geq 0,\ z_{kn} \geq 0.
\end{align*}
$U$, $Z$ can be minimised by an EM algorithm similar to the one for the Gaussian Mixture Model.

\subsubsection{Quadratic NMF}
So far: pLSI as a specific NMF for document indexing. NMF is a more general concept though. 

Consider a non-negative $X$ and a quadratic cost function as in $K$-means:
\begin{align*}
     \min_{U,Z}& J(U,Z) = {1\over 2} \norm{X-UZ}_F^2\\
     \text{s.t.}& u_{dk} \in [0,\infty)\ \forall d,k\\
     &z_{kn} \in [0,\infty)\ \forall k, n.
\end{align*}


\paragraph{Algorithm} The NMF algorithm is similar to the $K$-means algorithm in that it alternates between two update steps:
\begin{algorithmic}
\STATE $U \gets rand(D,K)$
\STATE $Z \gets rand(K,N)$
\FOR {$i= 1:maxiter$}
    \STATE Update Factors $U$: $u_{dk} \gets u_dk {(XZ^T)_{dk}\over (UZZ^T)_{dk}}.$
    \STATE Update coefficients $Z$: $z_{kn} \gets z_{kn} {(U^TX)_{kn}\over (U^TUZ)_{kn}}.$

\ENDFOR
\end{algorithmic}
This leads to $X\approx UZ$ when $K<N$ or possibly $X=UZ$ for $N\leq K$.

\subsubsection{Semi-NMF (for Quadratic Cost)}
When we relax the non-negativity assumption on $U$ this leads to the so called
\emph{Semi Non-Negative Matrix Factorisation}. The semi-NMF algorithm is similar to the $K$-means algorithm and to the NMF algorithm. It also alternates between two update steps:
\begin{algorithmic}
\STATE Update $U$: $U=XZ^T (ZZ^T)^{-1}$.
\STATE Update $Z$: $z_{kn} \gets z_{kn} \sqrt{{(U^T X)_{kn}^T +[(U^T U)^- Z]_{kn} \over (U^TX)_{kn}^- + [(U^TU)^+ Z]_{kn}}}$.
\end{algorithmic}
Here: $a_{ij}^+ := \max(0,a_{ij})$ and $a_{ij}^- := \min(0,a_{ij})$.$u_{dk} \gets u_dk {(XZ^T)_{dk}\over (UZZ^T)_{dk}}.$

\subsubsection{$K$-means Clustering Theorem}
$Z$-orthoganl ($ZZ^T = I$) semi-NMF is equivalent to $K$-means clustering:
\begin{align*}
    \min_{U,Z}\ &J(U,Z) ={1\over 2} \norm{X-UZ}_F^2.\\
    \text{s.t.}\ &Z\in \{0,1\}^{K\times N}\\
    &\sum_k z_{kn} = 1.
\end{align*}
is equivalent to:
\begin{align*}
    \min_{U,Z}\ &J(U,Z) ={1\over 2} \norm{X-UZ}_F^2.\\
    \text{s.t.}\ & ZZ^T = I,Z \geq 0.
\end{align*}

If $Z$ is not restricted to be orthogonal, semi-NMF is equivalent to a \emph{soft} $K$-means clustering (not GMM!).

\subsubsection{Convex-NMF}
For interpretability reasons we can impose the constraint that $U$ lies within the column space of $X$:
\begin{align*}
    u_k = w_{1k} x_1 + \ldots + w_{nk} x_n,\qquad \qquad U=XW,
\end{align*}
with $W\in \R_+^{N\times K}$.

So $X$ is factorized in the following way:
\begin{align*}
    X\approx UZ = XWZ.
\end{align*}

$U$ is a \emph{convex combination} of input data. The convexity of the data combination \emph{does not lead} to a convex optimisation problem. The algorithm for convex-NMF follows an analogous iterative update procedure. 

Convex-NMF factors $U$ are naturally \emph{sparse}



 


























