\documentclass[a4paper]{scrartcl}

\usepackage{amsmath, amsthm, amssymb, amsfonts} 
\usepackage{german}
\usepackage[utf8]{inputenc}
\usepackage{multicol}  
%\usepackage[rflt]{floatflt}  
\usepackage{graphics}
\usepackage{epsfig} 
%\usepackage{qtree} 
\usepackage{textcomp}

\usepackage{listings} 
% \lstset{numbers=left, numberstyle=\tiny, numbersep=5pt} 
\lstset{
	basicstyle=\ttfamily\scriptsize\mdseries,
	keywordstyle=\bfseries\color{blue},
	identifierstyle=,
% 	stringstyle=\itshape\color{red},
	numbers=left,
	numberstyle=\tiny,
	stepnumber=10,
	breaklines=true,
	frame=none,
	showstringspaces=false,
	tabsize=4,
% 	backgroundcolor=\color{gray},
% 	morecomment=[s][\color{green}]{/+}{+/},
	commentstyle=\color{gray},	
	captionpos=b,
	float=htbp,
}


\usepackage{tbabstract}
\newcommand{\K}     {\mathbb{K}}

\author{Pascal Spörri}
\title{Matlab Commands\\Summary HS 2009}
\date{\today}

\begin{document}
\maketitle
Beware: This document is probably not CC-able, since I copied from a variety of sources.
\section{Matrix Operations}
	\subsection{Triu} 
		U = triu(X) returns the upper triangular part of $X$.
		U = triu(X,k) returns the element on and above the kth 
		diagonal of X. k = 0 is the main diagonal, k > 0 is above 
		the main diagonal, and k < 0 is below the main diagonal.
	\subsection{Tril}
		U = tril(X) Does the same like triu, but instead of returning the upper triangular part of $X$ it returns the lower part.
	\subsection{Meshgrid} 
		The meshgrid command is used to generate two arrays 
		containing the $x$- and $y$-coordinates at each position in 
		a rectilinear grid. For example [X,Y] = meshgrid(-5:1:5) 
		returns two $11 \times 11$ matrices - the $X$ matrix defines the 
		$x$-coordinates and the $Y$ matrix the $y$-coordinates at each 
		position in an $11 \times 11$ grid.
		
		[X,Y] = meshgrid(1:n,1:m)
		\begin{align*}
			X &= \left. \begin{pmatrix}
				1 & 2 & \ldots & n\\
				1 & 2 & \ldots & n\\
				\vdots & & \ddots & \vdots \\
				1 & & 2 & \ldots & n 
				\end{pmatrix} \right\} m-\text{ mal}\\
			Y &= \underbrace{\begin{pmatrix}
						1 & 1 & \ldots & 1\\
						2 & 2 & \ldots & 2\\
						\vdots & &\ddots &\vdots\\
						m & m & \ldots & m 
					\end{pmatrix}}_{n-\text{ mal}}
		\end{align*}
	\subsection{Kron} 
		Kron(A,B) will create a special Matrix, that has all possible products between the elements of A and B.
		\begin{align*}
		y = kron\left( \begin{pmatrix}
						1 & 2\\
						3 & 4
						\end{pmatrix}, \begin{pmatrix}
										5 & 6 \\
										7 & 8 
										\end{pmatrix} \right) =
			M \ast x = \begin{pmatrix} 5 &   6 &  10 &  12\\
										7 &   8 &  14 &  16\\
										15 &  18 &  20 &  24\\
										21 &  24 &  28 &  32
						\end{pmatrix} \ast x
		\end{align*}
		\paragraph{Complexity} $O(n^4)$
\section{Sparse Matrices (this section is not CC-ed since i copied from octave and matlab doc )}
			When solving linear equations involving sparse matrices Octave
			determines the means to solve the equation based on the type of the
			matrix as discussed in Sparse Linear Algebra.  Octave probes the
			matrix type when the div ($/$) or ldiv ($\backslash$) operator is first used with
			the matrix and then caches the type.  However the ``matrix\_type``
			function can be used to determine the type of the sparse matrix prior
			to use of the div or ldiv operators.

	\subsection{sparse(A)}
			\begin{lstlisting}[language=matlab]
				S = sparse(I, J, SV)
				S = sparse(I, J, S, M, N, 'unique')
				S = sparse(M, N)
			\end{lstlisting}
			Create a sparse matrix from the full matrix or row, column, value
			triplets.  If A is a full matrix, convert it to a sparse matrix
			representation, removing all zero values in the process.

			Given the integer index vectors I and J, a 1-by-`nnz' vector of
			real of complex values SV, overall dimensions M and N of the
			sparse matrix.  If M or N are not specified their
			values are derived from the maximum index in the vectors I and J
			as given by $M = max (I)$, $N = max (J)$

			Iif multiple values are specified with the same I, J
			indices, the corresponding values in S will be added.

			The following are all equivalent:
			\begin{lstlisting}[language=matlab]
				s = sparse (i, j, s, m, n)
				s = sparse (i, j, s, m, n, 'summation')
				s = sparse (i, j, s, m, n, 'sum')
			\end{lstlisting}
			Given the option ''unique``. if more than two values are specified
			for the same I, J indices, the last specified value will be used.
			\begin{lstlisting}[language=matlab]
				sparse(M, N)
			\end{lstlisting}
			is equivalent to
			\begin{lstlisting}[language=matlab]
				sparse ([], [], [], M, N, 0)\]
			\end{lstlisting}

			If any of SV, I or J are scalars, they are expanded to have a
			common size.
		\paragraph{Example} S = sparse(1:n,1:n,1) generates a sparse representation of the n-by-n identity matrix. The same S results from S = sparse(eye(n,n)), but this would also temporarily generate a full n-by-n matrix with most of its elements equal to zero.
	\subsection{spdiags (A)}
			\begin{lstlisting}[language=matlab]
				[B, C] = spdiags (A)
				B = spdiags (A, C)
				B = spdiags (V, C, A)
				B = spdiags (V, C, M, N)
			\end{lstlisting}
			A generalization of the function `diag'.  Called with a single
			input argument, the non-zero diagonals C of A are extracted.  With
			two arguments the diagonals to extract are given by the vector C.

			The other two forms of `spdiags' modify the input matrix by
			replacing the diagonals.  They use the columns of V to replace the
			columns represented by the vector C.  If the sparse matrix A is
			defined then the diagonals of this matrix are replaced.  Otherwise
			a matrix of M by N is created with the diagonals given by V.

			Negative values of C represent diagonals below the main diagonal,
     		and positive values of C diagonals above the main diagonal.

     		For example
			\begin{lstlisting}[language=matlab]
				spdiags(reshape (1:12, 4, 3), [-1 0 1], 5, 4)
			\end{lstlisting}
			\begin{align*}
	\Longrightarrow	\qquad	\begin{pmatrix}
								5& 10&  0&  0\\
								1&  6& 11&  0\\
								0&  2&  7& 12\\
								0&  0&  3&  8\\
								0&  0&  0&  4
							\end{pmatrix}
			\end{align*}
	\subsection{speye(M)}
			\begin{lstlisting}[language=matlab]
				Y = speye (M)
				Y = speye (M, N)
				Y = speye (SZ)
			\end{lstlisting}

			Returns a sparse identity matrix.  This is significantly more
			efficient than ''sparse (eye (M))`` as the full matrix is not
			constructed.

			Called with a single argument a square matrix of size M by M is
			created.  Otherwise a matrix of M by N is created.  If called with
			a single vector argument, this argument is taken to be the size of
			the matrix to create.

	\subsection{nnz (A)}
			Returns the number of non zero elements in A.
	\subsection{nonzeros (S)}
			Returns a vector of the non-zero values of the sparse matrix S.

\end{document}
